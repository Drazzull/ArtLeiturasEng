%% abtex2-modelo-artigo.tex, v-1.9.6 laurocesar
%% Copyright 2012-2016 by abnTeX2 group at http://www.abntex.net.br/ 
%%
%% This work may be distributed and/or modified under the
%% conditions of the LaTeX Project Public License, either version 1.3
%% of this license or (at your option) any later version.
%% The latest version of this license is in
%%   http://www.latex-project.org/lppl.txt
%% and version 1.3 or later is part of all distributions of LaTeX
%% version 2005/12/01 or later.
%%
%% This work has the LPPL maintenance status `maintained'.
%% 
%% The Current Maintainer of this work is the abnTeX2 team, led
%% by Lauro César Araujo. Further information are available on 
%% http://www.abntex.net.br/
%%
%% This work consists of the files abntex2-modelo-artigo.tex and
%% abntex2-modelo-references.bib
%%

% ------------------------------------------------------------------------
% ------------------------------------------------------------------------
% abnTeX2: Modelo de Artigo Acadêmico em conformidade com
% ABNT NBR 6022:2003: Informação e documentação - Artigo em publicação 
% periódica científica impressa - Apresentação
% ------------------------------------------------------------------------
% ------------------------------------------------------------------------

\documentclass[
	% -- opções da classe memoir --
	article,			% indica que é um artigo acadêmico
	11pt,				% tamanho da fonte
	oneside,			% para impressão apenas no recto. Oposto a twoside
	a4paper,			% tamanho do papel. 
	% -- opções da classe abntex2 --
	%chapter=TITLE,		% títulos de capítulos convertidos em letras maiúsculas
	%section=TITLE,		% títulos de seções convertidos em letras maiúsculas
	%subsection=TITLE,	% títulos de subseções convertidos em letras maiúsculas
	%subsubsection=TITLE % títulos de subsubseções convertidos em letras maiúsculas
	% -- opções do pacote babel --
	english,			% idioma adicional para hifenização
	brazil,				% o último idioma é o principal do documento
	sumario=tradicional
%	twocolumn
	]{abntex2}


% ---
% PACOTES
% ---

% ---
% Pacotes fundamentais 
% ---
\usepackage{lmodern}			% Usa a fonte Latin Modern
\usepackage[T1]{fontenc}		% Selecao de codigos de fonte.
\usepackage[utf8]{inputenc}		% Codificacao do documento (conversão automática dos acentos)
\usepackage{indentfirst}		% Indenta o primeiro parágrafo de cada seção.
\usepackage{nomencl} 			% Lista de simbolos
\usepackage{color}				% Controle das cores
\usepackage{graphicx}			% Inclusão de gráficos
\usepackage{microtype} 			% para melhorias de justificação
% ---
		
% ---
% Pacotes adicionais, usados apenas no âmbito do Modelo Canônico do abnteX2
% ---
\usepackage{lipsum}				% para geração de dummy text
% ---
		
% ---
% Pacotes de citações
% ---
\usepackage[brazilian,hyperpageref]{backref}	 % Paginas com as citações na bibl
\usepackage[alf]{abntex2cite}	% Citações padrão ABNT
% ---

% ---
% Configurações do pacote backref
% Usado sem a opção hyperpageref de backref
\renewcommand{\backrefpagesname}{Citado na(s) página(s):~}
% Texto padrão antes do número das páginas
\renewcommand{\backref}{}
% Define os textos da citação
\renewcommand*{\backrefalt}[4]{
	\ifcase #1 %
		Nenhuma citação no texto.%
	\or
		Citado na página #2.%
	\else
		Citado #1 vezes nas páginas #2.%
	\fi}%
% ---

% ---
% Informações de dados para CAPA e FOLHA DE ROSTO
% ---
\titulo{Influência da Leitura na Formação do Perfil Profissional do Engenheiro}
\autor{Arthur Boesing Bilibio \and José Luiz Moresco Kaszuba \and Wagner Casagrande}
\local{Brasil}
\data{2016}
% ---

% ---
% Configurações de aparência do PDF final

% alterando o aspecto da cor azul
\definecolor{blue}{RGB}{41,5,195}

% informações do PDF
\makeatletter
\hypersetup{
	%pagebackref=true,
	pdftitle={\@title},
	pdfauthor={\@author},
	pdfsubject={Modelo de artigo científico com abnTeX2},
	pdfcreator={Arthur Bilibio},
	pdfkeywords={abnt}{latex}{abntex}{abntex2}{atigo científico},
	colorlinks=true,       		% false: boxed links; true: colored links
	linkcolor=black,          	% color of internal links
	citecolor=black,        	% color of links to bibliography
	filecolor=black,	      	% color of file links
	urlcolor=black,
	bookmarksdepth=4
}
\makeatother
% --- 

% ---
% compila o indice
% ---
\makeindex
% ---

% ---
% Altera as margens padrões
% ---
\setlrmarginsandblock{3cm}{3cm}{*}
\setulmarginsandblock{3cm}{3cm}{*}
\checkandfixthelayout
% ---

% --- 
% Espaçamentos entre linhas e parágrafos 
% --- 

% O tamanho do parágrafo é dado por:
\setlength{\parindent}{1.3cm}

% Controle do espaçamento entre um parágrafo e outro:
\setlength{\parskip}{0.2cm}  % tente também \onelineskip

% Espaçamento simples
\SingleSpacing

% ----
% Início do documento
% ----
\begin{document}

% Seleciona o idioma do documento (conforme pacotes do babel)
%\selectlanguage{english}
\selectlanguage{brazil}

% Retira espaço extra obsoleto entre as frases.
\frenchspacing 

% ----------------------------------------------------------
% ELEMENTOS PRÉ-TEXTUAIS
% ----------------------------------------------------------

%---
%
% Se desejar escrever o artigo em duas colunas, descomente a linha abaixo
% e a linha com o texto ``FIM DE ARTIGO EM DUAS COLUNAS''.
% \twocolumn[    		% INICIO DE ARTIGO EM DUAS COLUNAS
%
%---
% página de titulo
\maketitle

% resumo em português
\begin{resumoumacoluna}
	Conforme a ABNT NBR 6022:2003, o resumo é elemento obrigatório, constituído de
	uma sequência de frases concisas e objetivas e não de uma simples enumeração
	de tópicos, não ultrapassando 250 palavras, seguido, logo abaixo, das palavras
	representativas do conteúdo do trabalho, isto é, palavras-chave e/ou
	descritores, conforme a NBR 6028. (\ldots) As palavras-chave devem figurar logo
	abaixo do resumo, antecedidas da expressão Palavras-chave:, separadas entre si por
	ponto e finalizadas também por ponto.
	
	\vspace{\onelineskip}
	
	\noindent
	\textbf{Palavras-chave}: latex. abntex. editoração de texto.
\end{resumoumacoluna}

% ]  				% FIM DE ARTIGO EM DUAS COLUNAS
% ---

% ----------------------------------------------------------
% ELEMENTOS TEXTUAIS
% ----------------------------------------------------------
\textual

% ----------------------------------------------------------
% Introdução
% ----------------------------------------------------------
%\section*{Introdução}
\section{INTRODUÇÃO}
\addcontentsline{toc}{section}{Introdução}
Por senso comum, sabe-se que os acadêmicos de cursos de engenharias tendem a ler menos do que os acadêmicos de outros cursos. A partir disso pretende-se definir quais os tipos de leituras influenciam a formação dos estudantes de engenharia da UNOESC (Universidade do Oeste de Santa Catarina).

Para que esse objetivo seja alcançado, é necessário, antes, conhecer quais os tipos textuais são mais atrativos aos estudantes dos cursos de engenharia, descobrir se os textos lidos têm influência na área de formação dos estudantes e analisar a influencia da leitura na sua formação.

Também, planeja-se descobrir qual o grau de leitura atual dos mesmos e qual a influência do hábito de leitura na sua formação como engenheiros. Sabe-se que o hábito da leitura propicia o desenvolvimento de textos bem estruturados e coesos, nesse sentido visa-se saber quais gêneros textuais os estudantes produzem.

O desenvolvimento deste trabalho foi dividido em Referencial Teórico, contendo as referências utilizadas para desenvolvimento deste, Análise e Interpretação dos Dados, cujo objetivo é analisar os gráficos com os resultados obtidos durante o desenvolvimento e Considerações Finais, com os resultados encontrados.

Para o desenvolvimento deste trabalho, foi aplicado um questionário para 118 alunos de diversos cursos de engenharia da UNOESC Campus Joaçaba. 

Procedimentos metodológicos podem estar na introdução.


% ----------------------------------------------------------
% Seção de explicações
% ----------------------------------------------------------
\section{DESENVOLVIMENTO}

\subsection{REFERENCIAL TEÓRICO}

%\subsection{PROCEDIMENTOS METODOLÓGICOS}

\subsection{ANÁLISE E INTERPRETAÇÃO DOS DADOS}

% ---
% Finaliza a parte no bookmark do PDF, para que se inicie o bookmark na raiz
% ---
\bookmarksetup{startatroot}% 
% ---

% ---
% Conclusão
% ---
\subsection{CONSIDERAÇÕES FINAIS}

% ----------------------------------------------------------
% ELEMENTOS PÓS-TEXTUAIS
% ----------------------------------------------------------
\postextual

% ---
% Título e resumo em língua estrangeira
% ---

% \twocolumn[    		% INICIO DE ARTIGO EM DUAS COLUNAS

% titulo em inglês
%\titulo{Reading Influence in the Formation of Work Profile of The Engineers}
%\emptythanks
%\maketitle

% resumo em português
%\renewcommand{\resumoname}{Abstract}
%\begin{resumoumacoluna}
%	\begin{otherlanguage*}{english}
%		According to ABNT NBR 6022:2003, an abstract in foreign language is a back
%		matter mandatory element.
%		
%		\vspace{\onelineskip}
%		
%		\noindent
%		\textbf{Keywords}: latex. abntex.
%	\end{otherlanguage*}  
%\end{resumoumacoluna}

% ]  				% FIM DE ARTIGO EM DUAS COLUNAS
% ---

% ----------------------------------------------------------
% Referências bibliográficas
% ----------------------------------------------------------
\bibliography{abntex2-modelo-references}

% ----------------------------------------------------------
% Glossário
% ----------------------------------------------------------
%
% Há diversas soluções prontas para glossário em LaTeX. 
% Consulte o manual do abnTeX2 para obter sugestões.
%
%\glossary

% ----------------------------------------------------------
% Apêndices
% ----------------------------------------------------------
% ---
% Inicia os apêndices
% ---
%\begin{apendicesenv}
%	% ----------------------------------------------------------
%	\chapter{Nullam elementum urna vel imperdiet sodales elit ipsum pharetra ligula
%		ac pretium ante justo a nulla curabitur tristique arcu eu metus}
%	% ----------------------------------------------------------
%	\lipsum[55-57]
%\end{apendicesenv}
% ---

% ----------------------------------------------------------
% Anexos
% ----------------------------------------------------------
%\cftinserthook{toc}{AAA}
% ---
% Inicia os anexos
% ---
%\anexos
%\begin{anexosenv}
%	% ---
%	\chapter{Cras non urna sed feugiat cum sociis natoque penatibus et magnis dis
%		parturient montes nascetur ridiculus mus}
%	% ---
%	
%	\lipsum[31]
%\end{anexosenv}

\end{document}
