%% abtex2-modelo-artigo.tex, v-1.9.6 laurocesar
%% Copyright 2012-2016 by abnTeX2 group at http://www.abntex.net.br/ 
%%
%% This work may be distributed and/or modified under the
%% conditions of the LaTeX Project Public License, either version 1.3
%% of this license or (at your option) any later version.
%% The latest version of this license is in
%%   http://www.latex-project.org/lppl.txt
%% and version 1.3 or later is part of all distributions of LaTeX
%% version 2005/12/01 or later.
%%
%% This work has the LPPL maintenance status `maintained'.
%% 
%% The Current Maintainer of this work is the abnTeX2 team, led
%% by Lauro César Araujo. Further information are available on 
%% http://www.abntex.net.br/
%%
%% This work consists of the files abntex2-modelo-artigo.tex and
%% abntex2-modelo-references.bib
%%

% ------------------------------------------------------------------------
% ------------------------------------------------------------------------
% abnTeX2: Modelo de Artigo Acadêmico em conformidade com
% ABNT NBR 6022:2003: Informação e documentação - Artigo em publicação 
% periódica científica impressa - Apresentação
% ------------------------------------------------------------------------
% ------------------------------------------------------------------------

\documentclass[
	% -- opções da classe memoir --
	article,			% indica que é um artigo acadêmico
	11pt,				% tamanho da fonte
	oneside,			% para impressão apenas no recto. Oposto a twoside
	a4paper,			% tamanho do papel. 
	% -- opções da classe abntex2 --
	%chapter=TITLE,		% títulos de capítulos convertidos em letras maiúsculas
	%section=TITLE,		% títulos de seções convertidos em letras maiúsculas
	%subsection=TITLE,	% títulos de subseções convertidos em letras maiúsculas
	%subsubsection=TITLE % títulos de subsubseções convertidos em letras maiúsculas
	% -- opções do pacote babel --
	english,			% idioma adicional para hifenização
	brazil,				% o último idioma é o principal do documento
	sumario=tradicional
%	twocolumn
	]{abntex2}


% ---
% PACOTES
% ---

% ---
% Pacotes fundamentais 
% ---
\usepackage{lmodern}			% Usa a fonte Latin Modern
\usepackage[T1]{fontenc}		% Selecao de codigos de fonte.
\usepackage[utf8]{inputenc}		% Codificacao do documento (conversão automática dos acentos)
\usepackage{indentfirst}		% Indenta o primeiro parágrafo de cada seção.
\usepackage{nomencl} 			% Lista de simbolos
\usepackage{color}				% Controle das cores
\usepackage{graphicx}			% Inclusão de gráficos
\usepackage{microtype} 			% para melhorias de justificação
\usepackage{placeins}	        % FLoatBarrie
% ---
		
% ---
% Pacotes adicionais, usados apenas no âmbito do Modelo Canônico do abnteX2
% ---
\usepackage{lipsum}				% para geração de dummy text
% ---
		
% ---
% Pacotes de citações
% ---
\usepackage[brazilian,hyperpageref]{backref}	 % Paginas com as citações na bibl
\usepackage[alf]{abntex2cite}	% Citações padrão ABNT
% ---

% ---
% Configurações do pacote backref
% Usado sem a opção hyperpageref de backref
\renewcommand{\backrefpagesname}{Citado na(s) página(s):~}
% Texto padrão antes do número das páginas
\renewcommand{\backref}{}
% Define os textos da citação
\renewcommand*{\backrefalt}[4]{
	\ifcase #1 %
		Nenhuma citação no texto.%
	\or
		Citado na página #2.%
	\else
		Citado #1 vezes nas páginas #2.%
	\fi}%
% ---

% ---
% Informações de dados para CAPA e FOLHA DE ROSTO
% ---
\titulo{Influência da Leitura na Formação do Perfil Profissional do Engenheiro}
\autor{Arthur Boesing Bilibio \and José Luiz Moresco Kaszuba \and Wagner Casagrande}
\local{Brasil}
\data{2016}
% ---

% ---
% Configurações de aparência do PDF final

% alterando o aspecto da cor azul
\definecolor{blue}{RGB}{41,5,195}

% informações do PDF
\makeatletter
\hypersetup{
	%pagebackref=true,
	pdftitle={\@title},
	pdfauthor={\@author},
	pdfsubject={Modelo de artigo científico com abnTeX2},
	pdfcreator={Arthur Bilibio},
	pdfkeywords={abnt}{latex}{abntex}{abntex2}{atigo científico},
	colorlinks=true,       		% false: boxed links; true: colored links
	linkcolor=black,          	% color of internal links
	citecolor=black,        	% color of links to bibliography
	filecolor=black,	      	% color of file links
	urlcolor=black,
	bookmarksdepth=4
}
\makeatother
% --- 

% ---
% compila o indice
% ---
\makeindex
% ---

% ---
% Altera as margens padrões
% ---
\setlrmarginsandblock{3cm}{3cm}{*}
\setulmarginsandblock{3cm}{3cm}{*}
\checkandfixthelayout
% ---

% --- 
% Espaçamentos entre linhas e parágrafos 
% --- 

% O tamanho do parágrafo é dado por:
\setlength{\parindent}{1.3cm}

% Controle do espaçamento entre um parágrafo e outro:
\setlength{\parskip}{0.2cm}  % tente também \onelineskip

% Espaçamento simples
\SingleSpacing

% ----
% Início do documento
% ----
\begin{document}

% Seleciona o idioma do documento (conforme pacotes do babel)
%\selectlanguage{english}
\selectlanguage{brazil}

% Retira espaço extra obsoleto entre as frases.
\frenchspacing 

% ----------------------------------------------------------
% ELEMENTOS PRÉ-TEXTUAIS
% ----------------------------------------------------------

%---
%
% Se desejar escrever o artigo em duas colunas, descomente a linha abaixo
% e a linha com o texto ``FIM DE ARTIGO EM DUAS COLUNAS''.
% \twocolumn[    		% INICIO DE ARTIGO EM DUAS COLUNAS
%
%---
% página de titulo
\maketitle

% resumo em português
\begin{resumoumacoluna}
	Conforme a ABNT NBR 6022:2003, o resumo é elemento obrigatório, constituído de
	uma sequência de frases concisas e objetivas e não de uma simples enumeração
	de tópicos, não ultrapassando 250 palavras, seguido, logo abaixo, das palavras
	representativas do conteúdo do trabalho, isto é, palavras-chave e/ou
	descritores, conforme a NBR 6028. (\ldots) As palavras-chave devem figurar logo
	abaixo do resumo, antecedidas da expressão Palavras-chave:, separadas entre si por
	ponto e finalizadas também por ponto.
	
	\vspace{\onelineskip}
	
	\noindent
	\textbf{Palavras-chave}: latex. abntex. editoração de texto.
\end{resumoumacoluna}

% ]  				% FIM DE ARTIGO EM DUAS COLUNAS
% ---

% ----------------------------------------------------------
% ELEMENTOS TEXTUAIS
% ----------------------------------------------------------
\textual

% ----------------------------------------------------------
% Introdução
% ----------------------------------------------------------
%\section*{Introdução}
\section{INTRODUÇÃO}
\addcontentsline{toc}{section}{Introdução}
Por senso comum, sabe-se que os acadêmicos de cursos de engenharias tendem a ler menos do que os acadêmicos de outros cursos. A partir disso pretende-se definir quais os tipos de leituras influenciam a formação dos estudantes de engenharia da UNOESC (Universidade do Oeste de Santa Catarina).

Para que esse objetivo seja alcançado, é necessário, antes, conhecer quais os tipos textuais são mais atrativos aos estudantes dos cursos de engenharia, descobrir se os textos lidos têm influência na área de formação dos estudantes e analisar a influência da leitura na sua formação.

Também, planeja-se descobrir qual o grau de leitura atual dos mesmos e qual a influência do hábito de leitura na sua formação como engenheiros. Sabe-se que o hábito da leitura propicia o desenvolvimento de textos bem estruturados e coesos, nesse sentido visa-se saber quais gêneros textuais os estudantes produzem.

O desenvolvimento deste trabalho foi dividido em Referencial Teórico, contendo as referências utilizadas para desenvolvimento deste, Análise e Interpretação dos Dados, cujo objetivo é analisar os gráficos com os resultados obtidos durante o desenvolvimento e Considerações Finais, com os resultados encontrados.

Para o desenvolvimento deste trabalho, foi aplicado um questionário para 118 alunos de diversos cursos de engenharia da UNOESC Campus Joaçaba. 

Procedimentos metodológicos podem estar na introdução.


% ----------------------------------------------------------
% Seção de explicações
% ----------------------------------------------------------
\subsection{REFERENCIAL TEÓRICO}
"A importância do presente estudo é gerar conhecimento a partir da percepção de que nossos alunos leem pouco e que isso dificulta o aprendizado e a capacidade de comunicação. Se tal percepção mostrar-se verdadeira, é de se supor que os egressos carreguem graves deficiências para sua vida, não só profissional." \cite[p.9.132]{habitosleit}

Os autores \citeonline{FichamentoBili} explicam que, o estereótipo do aluno de engenharia - de que o mesmo não está interessado nas matérias da área de humanas - está equivocado, pois os mesmos demonstram um comportamento totalmente divergente dessa "afirmação".

A importância da leitura está muito além de sua utilização como ferramenta do conhecimento ou da comunicação. Ela possui a capacidade de transformar uma pessoa, através da capacidade do usuário de dar sentido aos sinais e compreendê-los.

A falta do hábito de leitura por parte dos alunos, aliada à falta de incentivo à leitura nas escolas [...] "parece ser um fenômeno não apenas brasileiro, mas global. Ela é fruto da emergência da sociedade imagética. Tal conceituação desta, refere-se às imagens no sentido de que estímulos captados pela visão estão – associados ou não a outros – são os difusores das simbologias." \cite[p.9.132]{habitosleit}

O indivíduo passa a consumir informações imediatas através das imagens, não exigindo assim esforços ou responsabilidades. Compromissos duradouros não se adequam a tal indivíduo.

Quando o aluno entra no ensino superior, o mesmo não está com o vocabulário “correto”, porém, o mesmo é lapidado conforme o passar dos anos, de modo que a instituição acabe tornando o mesmo letrado. \cite{FichamentoBili}

Para identificar os hábitos de leituras entre os estudantes da UNESP, foi utilizado um questionário composto por 30 questões.
Quanto à idade, os entrevistados têm entre 17 e 34 anos. A média de livros lidos no ano de 2005 por estes estudantes foi de dois livros. 

"Quando perguntados sobre como percebe a importância da leitura para a formação profissional, obteve-se o maior consenso dentre todas as questões: 95,4\% consideram que ela é "muito importante", apenas 3,8\% consideram "pouco importante" e 0,8\%, "nada importante". O baixo nível de leitura é, em parte, reconhecido pelos estudantes, pois 44\% consideram que "está aquém" do que seria desejável para sua formação; 31\% acham que é adequado e 25\% não sabem avaliar." \cite[p.9.141]{habitosleit}

Segundo \citeonline{FichamentoBili}, para que os estudantes de engenharia obtenham o interesse desejado nas matérias da língua materna, é imprescindível o estudo das turmas que estarão cursando essas matérias, visto que o perfil de cada turma é diferente, com conhecimentos, dificuldades diversas e interesses diversos, de modo que seja possível fazer uma aula dinâmica.

\citeonline{habitosleit} concluem que os "dados coletados e tratados confirmam a percepção motivadora do estudo: a situação da leitura entre os estudantes é extremamente problemática".

Os alunos se sentem desinteressados nas matérias, pois segundo os professores, os mesmos não vem a real necessidade de aprender a escrita correta durante o decorrer das matérias \cite{FichamentoBili}.

"A aplicação de técnicas e de ferramentas que visem aumentar e ampliar o hábito de leitura certamente terão um papel importante. Mas é fundamental reafirmar que não se trata de uma questão meramente técnica. A quantidade, qualidade e a reflexão na leitura estão indissoluvelmente ligadas à forma de viver". \cite[p.9.143]{habitosleit}

Propõe-se então, promover cada vez mais o ensino como pesquisa. "Isso significaria a superação dos métodos de ensino baseados no repasse de conhecimentos desde o professor com destino ao aluno. Centrar o ensino na pesquisa. Neste caminho, a formulação de problemas seria muito mais importante que a resolução de questões já prontas. Haveria uma constante renovação e elaboração de seus conhecimentos novos, em que o estudante seria, ao mesmo tempo, um pesquisador e um aprendiz." \cite[p.9.143]{habitosleit}

Conforme \citeonline{FichamentoBili}, por mais que a pessoa saiba que está redigindo um texto incorretamente, a mesma o faz pois não é uma situação crítica, algumas pessoas têm em mente, de que apenas textos para superiores e afins é que devem ser escritos corretamente, os demais não possuem essa necessidade, de modo que não é necessário tempo para corrigí-lo.  


\section{DESENVOLVIMENTO}
Para a obtenção dos dados, foi analisado um grupo com 118 acadêmicos dos diversos cursos de engenharia ofertados pela instituição Unoesc (Universidade do Oeste de Santa Catarina) Campii de Joaçaba.

Conforme análise do questionário aplicado, identifica-se que 40,7\% dos entrevistados são do sexo feminino, enquanto 59,3\% são do sexo masculino.
Percebe-se que a maioria dos acadêmicos dos cursos de engenharia questionados são do sexo masculino. Estas informações podem ser observadas na \figurename{ \ref{sexo}}.

\begin{figure}[h]  
	\begin{center} 
		\begin{center}
			\changecaptionwidth 
			\captionwidth{13.5cm} %posicionamento da legenda
			\caption{\label{sexo} Sexo}
		 	{\includegraphics[scale=0.8]{imagens/sexo}}
			\fonte{o autor.}
		\end{center}
	\end{center}
\end{figure}
\FloatBarrier


Foram identificados que 72\% alunos estão entre 18 e 22 anos, 22\% estão entre 23 e 27 anos, 2,5\% estão entre 28 e 32 anos, 1,7\% estão entre 33 e 36 anos e 1,7\% estão acima de 37 anos. Estas informações podem ser observadas na \figurename{ \ref{idade}}.

\begin{figure}[h]  
	\begin{center} 
		\begin{center}
			\changecaptionwidth 
			\captionwidth{13.5cm} %posicionamento da legenda
			\caption{\label{idade} Idade}
			{\includegraphics[scale=0.8]{imagens/idade}}
			\fonte{o autor.}
		\end{center}
	\end{center}
\end{figure}
\FloatBarrier


A partir do grupo questionado, 27,1\% alunos são do curso de Engenharia de Computação, 46,6\% do curso de Engenharia Civil, 15,3\% do curso de Engenharia Elétrica, 8,5\% do curso de Engenharia Química e 2,5\% de Engenharia de Produção Mecânica. Os cursos de Engenharia Mecânica e Engenharia de Produção não foram questionados. Estas informações podem ser observadas na \figurename{ \ref{curso}}.

\begin{figure}[h]  
	\begin{center} 
		\begin{center}
			\changecaptionwidth 
			\captionwidth{13.5cm} %posicionamento da legenda
			\caption{\label{curso} Curso que Frequenta}
			{\includegraphics[scale=0.8]{imagens/curso}}
			\fonte{o autor.}
		\end{center}
	\end{center}
\end{figure}
\FloatBarrier


A partir da amostra, nota-se que 38,1\% dos estudantes trabalham na área de formação do curso que frequentam, 15,3\% trabalham fora da área de formação, 43,2\% são apenas estudantes e os outros 3,4\% responderam "Outros". Estas informações podem ser observadas na \figurename{ \ref{ocupacao}}.

\begin{figure}[h]  
	\begin{center} 
		\begin{center}
			\changecaptionwidth 
			\captionwidth{13.5cm} %posicionamento da legenda
			\caption{\label{ocupacao} Em relação a sua ocupação}
			{\includegraphics[scale=0.8]{imagens/ocupacao}}
			\fonte{o autor.}
		\end{center}
	\end{center}
\end{figure}
\FloatBarrier


75,2\% dos acadêmicos informam que a capacidade de conferir precisão e objetividade a informação são os elementos mais importantes para o âmbito profissional, 11,1\% informaram que a correção gramatical é importante, para 12\% dos entrevistados a padronização gramatical é um ponto forte, enquanto 1,7\% acham que outros elementos, além dos citados, são mais importantes para o ambiente profissional. Estas informações podem ser observadas na \figurename{ \ref{elementos}}.

\begin{figure}[h]  
	\begin{center} 
		\begin{center}
			\changecaptionwidth 
			\captionwidth{13.5cm} %posicionamento da legenda
			\caption{\label{elementos} Elementos da escrita são importantes para o âmbito profissional}
			{\includegraphics[scale=0.8]{imagens/elementosescrita}}
			\fonte{o autor.}
		\end{center}
	\end{center}
\end{figure}
\FloatBarrier


16,9\% informaram que os trabalhos acadêmicos são os seus textos mais escritos, 54,2\% informaram que os e-mails e relatórios são os gêneros mais escritos, 27,1\% responderam que são os generos digitais (redes sociais e blogs), enquanto 1,7\% responderam que escrevem outros tipos de textos. Estas informações podem ser observadas na \figurename{ \ref{genero}}.

\begin{figure}[h]  
	\begin{center} 
		\begin{center}
			\changecaptionwidth 
			\captionwidth{13.5cm} %posicionamento da legenda
			\caption{\label{genero} Quanto a escrita, qual o gênero textual você escreve com mais frequência}
			{\includegraphics[scale=0.8]{imagens/generotextual}}
			\fonte{o autor.}
		\end{center}
	\end{center}
\end{figure}
\FloatBarrier


37,3\% dos acadêmicos informaram que gostam de ler, 14,4\% não gostam de ler, enquanto 48,3\% responderam que as vezes gostam de ler. Estas informações podem ser observadas na \figurename{ \ref{gosta}}.

\begin{figure}[h]  
	\begin{center} 
		\begin{center}
			\changecaptionwidth 
			\captionwidth{13.5cm} %posicionamento da legenda
			\caption{\label{gosta} Você gosta de ler?}
			{\includegraphics[scale=0.8]{imagens/gostaler}}
			\fonte{o autor.}
		\end{center}
	\end{center}
\end{figure}
\FloatBarrier


Em um nível de 1 a 10, 2,5\% deram a nota 1, 3,4\% a nota 4, 5,1\% a nota 3, 5,9\% a nota 4, 14,4\% a nota 5, 16,1\% a nota 6, 17,8\% a nota 7, 19,5\% a nota 8, 5,9\% a nota 9 e 9,3\% a nota 10. Estas informações podem ser observadas na \figurename{ \ref{atraido}}.

\begin{figure}[h]  
	\begin{center} 
		\begin{center}
			\changecaptionwidth 
			\captionwidth{13.5cm} %posicionamento da legenda
			\caption{\label{atraido} Quão atraído você se sente pela leitura?}
			{\includegraphics[scale=0.8]{imagens/atraido}}
			\fonte{o autor.}
		\end{center}
	\end{center}
\end{figure}
\FloatBarrier


Gráfico "Quantos livros leu no último ano?":
13,6\% responderam que leram apenas 1 livro, 39,8\% leram até 3 livros, 16,1\% leram de 4 a 6 livros, 11,9\% leram de 7 a 10 livros, 0,8\% leram de 11 a 15 livros, 3,4\% leram mais que 15 livros e 14,4\% não leram nenhum livro. Estas informações podem ser observadas na \figurename{ \ref{livroslidos}}.

\begin{figure}[h]  
	\begin{center} 
		\begin{center}
			\changecaptionwidth 
			\captionwidth{13.5cm} %posicionamento da legenda
			\caption{\label{livroslidos} Quantos livros leu no último ano?}
			{\includegraphics[scale=0.8]{imagens/livroslidos}}
			\fonte{o autor.}
		\end{center}
	\end{center}
\end{figure}
\FloatBarrier


%\subsection{PROCEDIMENTOS METODOLÓGICOS}

\subsection{ANÁLISE E INTERPRETAÇÃO DOS DADOS}
Conforme pode-se analisar nos dados obtidos, a maioria dos estudantes 72\%, possuem entre 18 e 22 anos, sendo que esta é uma idade que os mesmos estão iniciando sua vida acadêmica e não sentem necessidade de ter uma boa escrita e um hábito de leitura frequente. Outro ponto interessante é que a maioria dos estudantes, 43,2\% estão apenas estudando, o que faz com que os mesmos tenham maior tempo para se dedicar a leitura, porém, a maioria ainda se sente pouco atraída pela leitura.


% ---
% Finaliza a parte no bookmark do PDF, para que se inicie o bookmark na raiz
% ---
\bookmarksetup{startatroot}% 
% ---

% ---
% Conclusão
% ---
\subsection{CONSIDERAÇÕES FINAIS}

% ----------------------------------------------------------
% ELEMENTOS PÓS-TEXTUAIS
% ----------------------------------------------------------
\postextual

% ---
% Título e resumo em língua estrangeira
% ---

% \twocolumn[    		% INICIO DE ARTIGO EM DUAS COLUNAS

% titulo em inglês
%\titulo{Reading Influence in the Formation of Work Profile of The Engineers}
%\emptythanks
%\maketitle

% resumo em português
%\renewcommand{\resumoname}{Abstract}
%\begin{resumoumacoluna}
%	\begin{otherlanguage*}{english}
%		According to ABNT NBR 6022:2003, an abstract in foreign language is a back
%		matter mandatory element.
%		
%		\vspace{\onelineskip}
%		
%		\noindent
%		\textbf{Keywords}: latex. abntex.
%	\end{otherlanguage*}  
%\end{resumoumacoluna}

% ]  				% FIM DE ARTIGO EM DUAS COLUNAS
% ---

% ----------------------------------------------------------
% Referências bibliográficas
% ----------------------------------------------------------
\bibliography{abntex2-modelo-references}

% ----------------------------------------------------------
% Glossário
% ----------------------------------------------------------
%
% Há diversas soluções prontas para glossário em LaTeX. 
% Consulte o manual do abnTeX2 para obter sugestões.
%
%\glossary

% ----------------------------------------------------------
% Apêndices
% ----------------------------------------------------------
% ---
% Inicia os apêndices
% ---
%\begin{apendicesenv}
%	% ----------------------------------------------------------
%	\chapter{Nullam elementum urna vel imperdiet sodales elit ipsum pharetra ligula
%		ac pretium ante justo a nulla curabitur tristique arcu eu metus}
%	% ----------------------------------------------------------
%	\lipsum[55-57]
%\end{apendicesenv}
% ---

% ----------------------------------------------------------
% Anexos
% ----------------------------------------------------------
%\cftinserthook{toc}{AAA}
% ---
% Inicia os anexos
% ---
%\anexos
%\begin{anexosenv}
%	% ---
%	\chapter{Cras non urna sed feugiat cum sociis natoque penatibus et magnis dis
%		parturient montes nascetur ridiculus mus}
%	% ---
%	
%	\lipsum[31]
%\end{anexosenv}

\end{document}
